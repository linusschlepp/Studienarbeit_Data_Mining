
% !TEX root = ../Vorlage.tex

\subsection{Modularisierung}
\LaTeX \ bietet die Möglichkeit der Modularisierung. Das heißt, dass Kapitel als Includes zu einem gesamten Dokument zusammengeführt werden können. Dabei werden alle Einstellungen in einem Hauptdokument vorgenommen und die Inhalte des Dokuments über den Befehl $\backslash include$ eingebunden. Die Includes definieren nur Dokumenteninhalt und verweisen auf ihr Hauptdokument. Als Beispiel wird dieser Punkt als Include eingebunden. Es muss dabei beachtet werden, dass nach jedem Include ein Seitenumbruch erfolgt. Daher wird die Verwendung von Includes nur für Sections empfohlen.
